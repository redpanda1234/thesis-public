\documentclass[border=1pt]{standalone}
\usepackage{tikz}
\usepackage{tikz-3dplot}
\usetikzlibrary{intersections, calc}
\usetikzlibrary{fpu}

\begin{document}

\pgfmathsetmacro{\phirot}{40}
\pgfmathsetmacro{\thetarot}{-40}
\tdplotsetmaincoords{\phirot}{\thetarot}

\begin{tikzpicture}[every node/.style={scale=1.5}, tdplot_main_coords]

  % Radius of the ball
  \pgfmathsetmacro{\r}{2*sqrt(2)}

  % Use a 45-45-90 triangle for the edges
  \coordinate (side1end) at (-2, -2, 0);
  \coordinate (side2end) at (2, -2, 0);

  \draw[name path=edges] (side1end) -- (0,0,0) -- (side2end);

  % Extend them past our epsilon ball with dotted segments
  \coordinate (side1ext) at (-3, -3,0);
  \coordinate (side2ext) at (3, -3,0);

  \draw[dashed] (side1end) -- (side1ext);
  \draw[dashed] (side2end) -- (side2ext);

  \def\drawgarbage#1#2#3{

    \pgfmathsetmacro{\t}{#1}
    \pgfmathsetmacro{\tredstop}{#2}
    \pgfmathsetmacro{\rgarbage}{#3}

    \pgfmathsetmacro{\tstart}{\t + 225}
    \pgfmathsetmacro{\tend}{225-3*\t}

    \pgfmathsetmacro{\xstart}{\r * cos(\tstart)}
    \pgfmathsetmacro{\ystart}{\r * sin(\tstart)}

    \pgfmathsetmacro{\xend}{\r * cos(\tend)}
    \pgfmathsetmacro{\yend}{\r * sin(\tend)}


    % \pgfset{fpu/output format=fixed}
    \pgfmathsetmacro{\xmid}{(\xstart+\xend)/2}
    \pgfmathsetmacro{\ymid}{(\ystart+\yend)/2}


    \path[name path=line] (\xstart, \ystart,0) -- (\xend, \yend,0);


    \node[
    name intersections={of=edges and line, total=\mytot}
    ] (myint) at (intersection-\mytot) {};


    \pgfgetlastxy{\xint}{\yint}; % Extract the coordinates of A


    % Length of the line between xstart and xend
    \pgfmathsetmacro{\linelen}{sqrt((\xstart-\xend)^2 + (\ystart-\yend)^2)}
    \pgfmathsetmacro{\localr}{\linelen/2}

    % Length of the y displacement between them
    \pgfmathsetmacro{\ydiff}{abs(\ystart - \yend)}

    % Calculate the rotated coordinate plane angle
    \pgfmathsetmacro{\rotby}{acos(abs(\ystart - \yend)/\linelen)}



    % We need this bs to get the standard x,z,y lengths
    \pgfpointxyz{1}{1}{1}
    \pgfgetlastxy{\xstd}{\ystd}; % Extract the coordinates
    \pgfmathsetmacro{\xint}{\xint/\xstd}
    \pgfmathsetmacro{\yint}{\yint/\ystd}
    % \node () at (0,0,4) {\xstd, \ystd};
    % \node () at (0,0,1) {\xmid, \ymid};
    % \node () at (0,0,2) {\xint, \yint};
    % \node () at (0,0,3) {\localr};

    \pgfmathsetmacro{\garbage}{1}

    % \pgfmathsetmacro{\tsfrac}{abs(\xmid - \xint)/(\localr)}



    % \pgfmathsetmacro{\thetastop}{asin(\tsfrac)}
    % \pgfmathsetmacro{\thetastop}{acos(abs(\xmid - \xint)/\linelen)}

    % \node[above left] () at (\xend, \yend) {\tiny \tsfrac};
    % \node[above left] () at (\xend, \yend) {\tiny \thetastop};

    \node (endpt) at (\xend, \yend, 0) {} ;

    \tdplotsetrotatedcoordsorigin{(endpt)}
    \tdplotsetrotatedcoords{\rotby}{90}{0}

    \begin{scope}
      \pgfmathsetmacro{\dropbyx}{-1*\rgarbage*\localr*sin(\tredstop)}
      \fill[
          tdplot_rotated_coords,
          fill=red!70!white,
          fill opacity=.1
      ]
      (0, 0, 0) arc (90:\tredstop:\localr)
      node[inner sep=0pt, outer sep=0pt] (crap) {}
      -- ++(\dropbyx, 0, 0) node[inner sep=0pt, outer sep=0pt] (crappier) {};

      \draw[tdplot_rotated_coords, red!70!white, opacity=.5] (0, 0, 0)
      arc (90:\tredstop:\localr);

      % \fill[tdplot_rotated_coords, fill=blue!70!white, fill
      % opacity=.1] (crappier) -- (crap) arc (\tredstop:270:\localr) -- (crappier);
      % \draw[tdplot_rotated_coords, blue!70!white, opacity=.5]
      % (crappier) -- (crap) arc (\tredstop:270:\localr) -- (crappier);
    \end{scope}

    \pgfmathsetmacro{\newx}{-\xend}
    \pgfmathsetmacro{\newy}{\yend}
    \node (newend) at (\newx, \newy, 0) {} ;
    \tdplotsetrotatedcoordsorigin{(newend)}
    \tdplotsetrotatedcoords{-\rotby}{90}{0}
    \begin{scope}
      \pgfmathsetmacro{\dropbyx}{-1*\rgarbage*\localr*sin(\tredstop)}
      \fill[
          tdplot_rotated_coords,
          fill=red!70!white,
          fill opacity=.1
      ]
      (0, 0, 0) arc (90:\tredstop:\localr)
      node[inner sep=0pt, outer sep=0pt] (crap) {}
      -- ++(\dropbyx, 0, 0) node[inner sep=0pt, outer sep=0pt] (crappier) {};

      \draw[tdplot_rotated_coords, red!70!white, opacity=.5] (0, 0, 0)
      arc (90:\tredstop:\localr);

      % \fill[tdplot_rotated_coords, fill=blue!70!white, fill
      % opacity=.1] (crappier) -- (crap) arc (\tredstop:270:\localr) -- (crappier);
      % \draw[tdplot_rotated_coords, blue!70!white, opacity=.5]
      % (crappier) -- (crap) arc (\tredstop:270:\localr) -- (crappier);
    \end{scope}
  }


  \def\drawbluegarbage#1#2#3{

    \pgfmathsetmacro{\t}{#1}
    \pgfmathsetmacro{\tredstop}{#2}
    \pgfmathsetmacro{\rgarbage}{#3}

    \pgfmathsetmacro{\tstart}{\t + 225}
    \pgfmathsetmacro{\tend}{225-3*\t}

    \pgfmathsetmacro{\xstart}{\r * cos(\tstart)}
    \pgfmathsetmacro{\ystart}{\r * sin(\tstart)}

    \pgfmathsetmacro{\xend}{\r * cos(\tend)}
    \pgfmathsetmacro{\yend}{\r * sin(\tend)}


    % \pgfset{fpu/output format=fixed}
    \pgfmathsetmacro{\xmid}{(\xstart+\xend)/2}
    \pgfmathsetmacro{\ymid}{(\ystart+\yend)/2}


    \path[name path=line] (\xstart, \ystart,0) -- (\xend, \yend,0);


    \node[
    name intersections={of=edges and line, total=\mytot}
    ] (myint) at (intersection-\mytot) {};


    \pgfgetlastxy{\xint}{\yint}; % Extract the coordinates of A


    % Length of the line between xstart and xend
    \pgfmathsetmacro{\linelen}{sqrt((\xstart-\xend)^2 + (\ystart-\yend)^2)}
    \pgfmathsetmacro{\localr}{\linelen/2}

    % Length of the y displacement between them
    \pgfmathsetmacro{\ydiff}{abs(\ystart - \yend)}

    % Calculate the rotated coordinate plane angle
    \pgfmathsetmacro{\rotby}{acos(abs(\ystart - \yend)/\linelen)}



    % We need this bs to get the standard x,z,y lengths
    \pgfpointxyz{1}{1}{1}
    \pgfgetlastxy{\xstd}{\ystd}; % Extract the coordinates
    \pgfmathsetmacro{\xint}{\xint/\xstd}
    \pgfmathsetmacro{\yint}{\yint/\ystd}

    \node (endpt) at (\xend, \yend, 0) {} ;

    \tdplotsetrotatedcoordsorigin{(endpt)}
    \tdplotsetrotatedcoords{\rotby}{90}{0}

    \begin{scope}
      \pgfmathsetmacro{\dropbyx}{-1*\rgarbage*\localr*sin(\tredstop)}
      \path[
          tdplot_rotated_coords,
      ]
      (0, 0, 0) arc (90:\tredstop:\localr)
      node[inner sep=0pt, outer sep=0pt] (crap) {}
      -- ++(\dropbyx, 0, 0) node[inner sep=0pt, outer sep=0pt] (crappier) {};

      \path[tdplot_rotated_coords] (0, 0, 0) arc (90:\tredstop:\localr);

      \fill[tdplot_rotated_coords, fill=blue!70!white, fill
      opacity=.1] (crappier) -- (crap) arc (\tredstop:270:\localr) -- (crappier);
      \draw[tdplot_rotated_coords, blue!70!white, opacity=.5]
      (crap) arc (\tredstop:270:\localr);
    \end{scope}

    \pgfmathsetmacro{\newx}{-\xend}
    \pgfmathsetmacro{\newy}{\yend}
    \node (newend) at (\newx, \newy, 0) {} ;
    \tdplotsetrotatedcoordsorigin{(newend)}
    \tdplotsetrotatedcoords{-\rotby}{90}{0}
    \begin{scope}
      \pgfmathsetmacro{\dropbyx}{-1*\rgarbage*\localr*sin(\tredstop)}
      \path[tdplot_rotated_coords]
      (0, 0, 0) arc (90:\tredstop:\localr)
      node[inner sep=0pt, outer sep=0pt] (crap) {}
      -- ++(\dropbyx, 0, 0) node[inner sep=0pt, outer sep=0pt] (crappier) {};

      \path[tdplot_rotated_coords] (0, 0, 0) arc (90:\tredstop:\localr);

      \fill[tdplot_rotated_coords, fill=blue!70!white, fill
      opacity=.1] (crappier) -- (crap) arc (\tredstop:270:\localr) -- (crappier);
      \draw[tdplot_rotated_coords, blue!70!white, opacity=.5]
      (crap) arc (\tredstop:270:\localr);
    \end{scope}
  }

  \foreach \t in {4, 8,..., 40, 45}{
    \pgfmathsetmacro{\tstart}{\t + 225}
    \pgfmathsetmacro{\tend}{225-3*\t}

    \pgfmathsetmacro{\xstart}{\r * cos(\tstart)}
    \pgfmathsetmacro{\ystart}{\r * sin(\tstart)}

    \pgfmathsetmacro{\xend}{\r * cos(\tend)}
    \pgfmathsetmacro{\yend}{\r * sin(\tend)}

    \path[name path=line] (\xstart, \ystart,0) -- (\xend, \yend,0);

    \node[
    name intersections={of=edges and line, total=\mytot}
    ] (myint) at (intersection-\mytot) {};

    \pgfgetlastxy{\xint}{\yint}; % Extract the coordinates of A


    % Length of the line between xstart and xend

    \draw[blue!70!white, opacity=.2] (\xstart, \ystart) -- (\xint,
    \yint);


    \draw[red!70!white, opacity=.2] (\xend, \yend) -- (\xint,
    \yint);


  }

  \foreach \t in {50, 54,..., 88}{
    \pgfmathsetmacro{\tstart}{\t + 225}
    \pgfmathsetmacro{\tend}{225-3*\t}

    \pgfmathsetmacro{\xstart}{\r * cos(\tstart)}
    \pgfmathsetmacro{\ystart}{\r * sin(\tstart)}

    \pgfmathsetmacro{\xend}{\r * cos(\tend)}
    \pgfmathsetmacro{\yend}{\r * sin(\tend)}

    \path[name path=line] (\xstart, \ystart,0) -- (\xend, \yend,0);

    \node[
    name intersections={of=edges and line, total=\mytot}
    ] (myint) at (intersection-\mytot) {};

    \pgfgetlastxy{\xint}{\yint}; % Extract the coordinates of A


    % Length of the line between xstart and xend

    \draw[blue!70!white, opacity=.4] (\xstart, \ystart) -- (\xint,
    \yint);


    \draw[red!70!white, opacity=.4] (\xend, \yend) -- (\xint,
    \yint);

  }


  \foreach \t/\tcrap/\rmultiplier in {
    % 44/182.5/23,
    % 43/184.25/13.5,
    % 42/185.5/10.4,
    % 41/187/8.15,
    40/188.5/6.65,
    % %
    % 39/190/5.7,
    % 38/191.25/5,
    % 37/192.5/4.5,
    36/193.75/4.1,
    % 35/194.75/3.8,
    % 34/195.75/3.55,
    % 33/196.8/3.3,
    32/197.8/3.1,
    % 31/198.7/2.95,
    % 30/199.5/2.8,
    % %
    % 29/200.25/2.7,
    28/201/2.6,
    % 27/201.8/2.5,
    % 26/202.5/2.4,
    % 25/203.15/2.35,
    24/203.5/2.3,
    % 23/204/2.25,
    % 22/204.7/2.175,
    % 21/205.5/2.1,
    20/205.7/2.1,
    % %
    % 19/206.2/2.05,
    % 18/206.7/1.975,
    % 17/207.15/1.95,
    16/207.6/1.9,
    % 15/208.0/1.875,
    % 14/208.3/1.85,
    % 13/208.5/1.825,
    12/208.7/1.825,
    % 11/209.2/1.825,
    %
    % 10/209.0/1.8,
    % 9/209.5/1.75,
    8/210/1.75,
    % 7/210.25/1.75,
    % 6/210.5/1.75,
    % 5/210.715/1.75,
    4/211/1.75,
    % 3/211.5/1.7,
  }{
    \drawgarbage{\t}{\tcrap}{\rmultiplier}
    % \pgfmathsetmacro{\othert}{90-\t}
  }


  \foreach \t/\tcrap/\rmultiplier in {
    % 45/180/90,
    % 45/180/23,
    % 44/182.5/23,
    % 43/184.25/13.5,
    % 42/185.5/10.4,
    % 41/187/8.15,
    40/188.5/6.65,
    % %
    % 39/190/5.7,
    % 38/191.25/5,
    % 37/192.5/4.5,
    36/193.75/4.1,
    % 35/194.75/3.8,
    % 34/195.75/3.55,
    % 33/196.8/3.3,
    32/197.8/3.1,
    % 31/198.7/2.95,
    % 30/199.5/2.8,
    % %
    % 29/200.25/2.7,
    28/201/2.6,
    % 27/201.8/2.5,
    % 26/202.5/2.4,
    % 25/203.15/2.35,
    24/203.5/2.3,
    % 23/204/2.25,
    % 22/204.7/2.175,
    % 21/205.5/2.1,
    20/205.7/2.1,
    % %
    % 19/206.2/2.05,
    % 18/206.7/1.975,
    % 17/207.15/1.95,
    16/207.6/1.9,
    % 15/208.0/1.875,
    % 14/208.3/1.85,
    % % 13/208.5/1.825,
    12/208.7/1.825,
    % 11/209.2/1.825,
    %
    % 10/209.0/1.8,
    % 9/209.5/1.75,
    8/210/1.75,
    % 7/210.25/1.75,
    % 6/210.5/1.75,
    % 5/210.715/1.75,
    4/211/1.75,
    % 3/211.5/1.7,
  }{
    \drawbluegarbage{\t}{\tcrap}{\rmultiplier}
    % \pgfmathsetmacro{\othert}{90-\t}
  }








  \pgfmathsetmacro{\t}{45}
  \pgfmathsetmacro{\tredstop}{180}


  \pgfmathsetmacro{\tstart}{\t + 225}
  \pgfmathsetmacro{\tend}{225-3*\t}

  \pgfmathsetmacro{\xstart}{\r * cos(\tstart)}
  \pgfmathsetmacro{\ystart}{\r * sin(\tstart)}

  \pgfmathsetmacro{\xend}{\r * cos(\tend)}
  \pgfmathsetmacro{\yend}{\r * sin(\tend)}


  % \pgfset{fpu/output format=fixed}
  \pgfmathsetmacro{\xmid}{(\xstart+\xend)/2}
  \pgfmathsetmacro{\ymid}{(\ystart+\yend)/2}


  \path[name path=line] (\xstart, \ystart,0) -- (\xend, \yend,0);


  \node[
  name intersections={of=edges and line, total=\mytot}
  ] (myint) at (intersection-\mytot) {};


  \pgfgetlastxy{\xint}{\yint}; % Extract the coordinates of A


  % Length of the line between xstart and xend
  \pgfmathsetmacro{\linelen}{sqrt((\xstart-\xend)^2 + (\ystart-\yend)^2)}
  \pgfmathsetmacro{\localr}{\linelen/2}

  % Length of the y displacement between them
  \pgfmathsetmacro{\ydiff}{abs(\ystart - \yend)}

  % Calculate the rotated coordinate plane angle
  \pgfmathsetmacro{\rotby}{acos(abs(\ystart - \yend)/\linelen)}



  % We need this bs to get the standard x,z,y lengths
  \pgfpointxyz{1}{1}{1}
  \pgfgetlastxy{\xstd}{\ystd}; % Extract the coordinates
  \pgfmathsetmacro{\xint}{\xint/\xstd}
  \pgfmathsetmacro{\yint}{\yint/\ystd}

  \pgfmathsetmacro{\garbage}{1}

  \node (endpt) at (\xend, \yend, 0) {} ;

  \tdplotsetrotatedcoordsorigin{(endpt)}
  \tdplotsetrotatedcoords{\rotby}{90}{0}

  \begin{scope}
    \fill[
    tdplot_rotated_coords,
    fill=red!70!white,
    fill opacity=.1
    ]
    (0, 0, 0) arc (90:\tredstop:\localr)
    node[inner sep=0pt, outer sep=0pt] (crap) {}
    -- ++(\r, 0, 0) node[inner sep=0pt, outer sep=0pt] (crappier) {};

    \draw[tdplot_rotated_coords, red!70!white, opacity=.5] (0, 0, 0)
    arc (90:\tredstop:\localr);

    \fill[tdplot_rotated_coords, fill=blue!70!white, fill
    opacity=.1] (crappier) -- (crap) arc (\tredstop:270:\localr) --
    (crappier) --cycle;
    \draw[tdplot_rotated_coords, blue!70!white, opacity=.5]
    % (crappier) --
    (crap) arc (\tredstop:270:\localr);%; -- (crappier) -- cycle;
  \end{scope}



  % Draw the circle
  \draw (side1end) -- (0,0,0) -- (side2end);
  % \node[fill=black, circle, inner sep=1pt] (x) at (0,0,0) {};

  \fill[red!70!white, opacity=.1] (side2end) arc (-45:225:\r) -- (0,0,0)
  -- (side2end);
  \draw[red, thick] (2, -2,0) arc(-45:225:\r);

  \fill[blue!70!white, opacity=.1] (side1end) -- (0,0,0) -- (side2end)
  arc (315:225:\r) --cycle;
  \draw[blue, thick] (-2, -2,0) arc(225:315:\r);

\end{tikzpicture}
\end{document}
