\documentclass[tikz,border=3.14pt]{standalone}
\usepackage{tikz-3dplot}

\begin{document}
\tdplotsetmaincoords{0}{0}
\begin{tikzpicture}[line cap=round, line join=round, tdplot_main_coords]
  \def\drawcyl{
    % Draw the vertical connecting lines for the cylinder
    % \draw (-\R, 0, \zmax) -- (-\R, 0, 0);
    % \draw (\R, 0, \zmax) -- (\R, 0, 0);

    % Bottom dashed lien part
    \draw plot[variable=\x,domain=180:360,samples=360]
    ({\R*cos(\x)},{\R*sin(\x)}, 0);

    % Undashed part
    \draw[dashed] plot[variable=\x,domain=0:180,samples=360]
    ({\R*cos(\x)},{\R*sin(\x)}, 0);

    % Top cap
    \draw plot[variable=\x,domain=0:360,samples=360]
    ({\R*cos(\x)},{\R*sin(\x)}, \zmax);


    % Top strand goes here
    \draw[blue]
    plot[variable=\x,domain=0:360,samples=360] ({\R*cos(\x)},{\R*sin(\x)}, \zts);


    % Bottom strand goes here
    \draw[red] plot[variable=\x,domain=0:360,samples=360]
    ({\R*cos(\x)},{\R*sin(\x)}, \zbs);

    \draw plot[variable=\x,domain=0:360,samples=360]
    ({\R*cos(\x)},{\R*sin(\x)}, .5*\zmax);

% Bottom dashed lien part
    \draw plot[variable=\x,domain=180:360,samples=360]
    ({\R*cos(\x)},{\R*sin(\x)}, 0);

    % Undashed part
    \draw[dashed] plot[variable=\x,domain=0:180,samples=360]
    ({\R*cos(\x)},{\R*sin(\x)}, 0);

  }

  \pgfmathsetmacro{\zmax}{10}

  \pgfmathsetmacro{\zts}{.75*\zmax}
  \pgfmathsetmacro{\zbs}{.25*\zmax}

  \pgfmathsetmacro{\Rout}{3}
  \pgfmathsetmacro{\Rin}{2}


  % Rotate to make the drawing of the crossing essentially trivial
  \tdplotsetrotatedcoords{-50}{0}{0}
  \begin{scope}[tdplot_rotated_coords]
    \draw[blue, thick] (-\Rout, 0, \zts) -- (\Rout, 0, \zts);
    \draw[red, thick] (0, -\Rout, \zbs) -- (0, \Rout, \zbs);


    \draw[red, thick] (0, -\Rout, 0) -- (0, \Rout, 0);
    \draw[white, line width=7pt] (-.5*\Rout, 0, 0) -- (.5*\Rout, 0, 0);
    \draw[blue, thick] (-\Rout, 0, 0) -- (\Rout, 0, 0);
  \end{scope}

  % Draw the circles and the vertical cylider lines
  \pgfmathsetmacro{\R}{\Rin}
  \drawcyl

  \foreach \t in {5, 25, ..., 365}{
    \pgfmathsetmacro{\mx}{\R*cos(\t)}
    \pgfmathsetmacro{\my}{\R*sin(\t)}

    % Draw the vertical connecting lines for the cylinder
    \draw[dotted] (\mx, \my, \zmax) -- (\mx, \my, 0);

  }


  \pgfmathsetmacro{\R}{\Rout}
  \drawcyl

  \foreach \t in {0, 180}{
    \pgfmathsetmacro{\mx}{\R*cos(\t)}
    \pgfmathsetmacro{\my}{\R*sin(\t)}

    % Draw the vertical connecting lines for the cylinder
    \draw[very thin] (\mx, \my, \zmax) -- (\mx, \my, 0);

  }




  % Now we darw all the oscillatory parts
  \pgfmathsetmacro{\ti}{135} % Initial theta
  \pgfmathsetmacro{\trange}{150}
  \pgfmathsetmacro{\Rs}{\Rout + .5}
  % \pgfmathsetmacro{\Rm}{\Rout}
  % \pgfmathsetmacro{\Rm}{\Rout}

  \foreach \n in {2, 3, ..., 40}{

    % This value of theta
    \pgfmathsetmacro{\theta}{\ti + \trange/\n}

    % Next value of n
    \pgfmathsetmacro{\nnext}{\n + 1}

    % How much to change theta by for the next one
    \pgfmathsetmacro{\dt}{\trange * (1/\n - 1/\nnext)}

    % How far in towards the center to go
    \pgfmathsetmacro{\Rin}{\Rin/\n}

    % Starting coordinates
    \pgfmathsetmacro{\xo}{\Rs * cos(\theta)}
    \pgfmathsetmacro{\yo}{\Rs * sin(\theta)}

    % Mid point on the inside
    \pgfmathsetmacro{\xm}{\Rin * cos(\theta - \dt/2)}
    \pgfmathsetmacro{\ym}{\Rin * sin(\theta - \dt/2)}

    % Final point going back out to the rim
    \pgfmathsetmacro{\xf}{\Rs * cos(\theta - \dt)}
    \pgfmathsetmacro{\yf}{\Rs * sin(\theta - \dt)}

    \draw[ultra thin] (\xo, \yo, .5*\zmax) -- (\xm, \ym, .5*\zmax) -- (\xf, \yf, .5*\zmax);
    \draw[ultra thin] (\xo, \yo, 0) -- (\xm, \ym, 0) -- (\xf, \yf, 0);
    % \pgfmathsetmacro{\ts}{\ti + \trange/\n}
    % \pgfmathsetmacro{\tm}{\ts + \dt/4}
    % \pgfmathsetmacro{\to}{\ts + 2*\dt/4}
    % \pgfmathsetmacro{\tf}{\ts + 3*\dt/4}

    % \pgfmathsetmacro{\xs}{\Rs*cos(\ts)}
    % \pgfmathsetmacro{\ys}{\Rs*sin(\ts)}

    % \pgfmathsetmacro{\xm}{\Rout*cos(\tm)}
    % \pgfmathsetmacro{\ym}{\Rout*sin(\tm)}

    % \pgfmathsetmacro{\xo}{\R*cos(\to)}
    % \pgfmathsetmacro{\yo}{\R*sin(\to)}

    % \pgfmathsetmacro{\xf}{\R*cos(\tf)}
    % \pgfmathsetmacro{\yf}{\R*sin(\tf)}

  }


  \pgfmathsetmacro{\inrx}{2*cos(-20)}
  \pgfmathsetmacro{\inry}{2*sin(-20)}
  \draw[dotted] (0,0) -- (\inrx, \inry) node[midway, below] {$\eta_0$};

  \pgfmathsetmacro{\outrx}{3*cos(10)}
  \pgfmathsetmacro{\outry}{3*sin(10)}
  \draw[dotted] (0,0) -- (\outrx, \outry) node[midway, above] {$\eta_1$};


\end{tikzpicture}
\end{document}
